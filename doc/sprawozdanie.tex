\documentclass{classrep.cls}
\usepackage[utf8]{inputenc}
\usepackage{color}
\usepackage{graphicx}

\DeclareUnicodeCharacter{00A0}{~}

\studycycle{Informatyka, studia dzienne, inż I st.}
\coursesemester{IV}

\coursename{Sztuczna inteligencja i systemy ekspertowe}
\courseyear{2023/2024}

\courseteacher{Krzysztof Lichy}
\coursegroup{poniedziałek, 12:15}

\author{
    \studentinfo{Mateusz Giełczyński}{247662} \and
    \studentinfo{Jakub Kubiś}{247712}
}

\title{Zadanie 1: Piętnastka}

\begin{document}
    \maketitle

    {\color{blue}
    Klasa sprawozdania wymaga następujących pakietów:
        \begin{itemize}
            \item zestaw klas autorstwa Marcina Woli\'nskiego \ppauza u\.zywana jest klasa
            \emph{mwart} (nazwa całego pakietu: \emph{mwcls}),
            \item pakiet \emph{polski},
            \item pakiet \emph{url}.
        \end{itemize}}


%{\color{blue}
%W preambule należy wykorzystać następujące niestandardowe polecenia, aby
%dostarczyć informacje wymagane do wygenerowania sprawozdania:
%\begin{itemize}
%    \item \verb+\studycycle{...}+ \ppauza tryb i rodzaj studiów,
%    \item \verb+\coursesemester{...}+ \ppauza semestr studiów,
%    \item \verb+\coursename{...}+ \ppauza nazwa przedmiotu,
%    \item \verb+\courseyear{...}+ \ppauza bieżący rok akademicki,
%    \item \verb+\courseteacher{...}+ \ppauza prowadzący laboratoria (proszę
%    wpisywać osobę, której oddaje się sprawozdanie, a nie kierownika
%    przedmiotu!),
%    \item \verb+\coursegroup{...}+ \ppauza identyfikator grupy ćwiczeniowej,
%    \item \verb+\studentinfo[e-mail]{imię i nazwisko}{nr albumu}+ \ppauza
%    polecenie wykorzystywane \emph{tylko} w obrębie polecenia \verb+\author+
%    służy do określenia imienia i nazwiska każdego z autorów, jak również
%    numeru albumu i opcjonalnie adresu e\dywiz mail; jeśli adres nie jest
%    podawany, należy pominąć argument opcjonalny.
%\end{itemize}}


    \section{Cel}
    {\color{blue}
    Celem zadania było napisanie programu znajdującego rozwiązanie układanki logicznej zwanej "Pietnastką" oraz zbadanie zasad działania
    algorytmów przeszukujących grafy.
    .}

    \section{Wprowadzenie}
    {\color{red}

    Piętnastka to układanka logiczna składająca się z kwadratowej
    planszy podzielonej na 16 kwadratów. Zawierających cyfry od 1 do 15 oraz jednym miejscem które zawsze pozostaje puste
    co pozwala na przesuwanie sąsiednich kafelków, celem piętnastki jest uporządkowanie kafelków tak aby uzyskać układ liczbowy
    gdzie liczby ułożone są w porządku rosnącym ,a ostatnie pole pozostaje puste.

    \begin{figure}
        \centering
        \includegraphics{img}
        \caption{}
        \label{fig:Rozwiązana piętnastka}
    \end{figure}

    Piętnastkę można zinterpretować jako graf, gdzie danym węzłem jest stan planszy. Natomiast krawędzie łączące węzły
    reprezentują możliwe ruchy pomiędzy stanami. Korzeniem w takim grafie będzie początkowy nieroziązany stan układanki.

    Inepretacja piętnastki jako grafu pozwala nam stosować metody przeszukiwania grafów do znajdywania rozwiązań układnaki.
    Zaimplementowane przez nas metody to:
        \begin{itemize}
            \item BFS (breadth-first search) - przeszukiwanie wszerz
                Metoda polega na przeszukaniu wszystkich węzłów na jednej głębokości przed zejściem na niższy poziom.
            \item DFS (depth-first search) - przeszukiwanie wgłąb
                Metoda polega na preszukaniu wszystkich krawędzi danego wierzchołka dopiero po rekurencyjnym przeszukaniu
                wszyskich krawędzi wierzchołków potomnych

            Metoda polega na przeszukiwaniu w głąb, oznacza to że zanim zbada kolejną krawedź pierwszego węzła
            dopiero gdy zbada wszystkie krawędzie kolejnego węzła
            \item A* (A-star)
                Metoda polega na wybieraniu najbardziej obiecujacego węzła na podstawie heurystyki.
                \begin{itemize}
                    \item Heurystyka Hamminga
                    \item Heurystyka Manhattana
                \end{itemize}
        \end{itemize}

    }

    \section{Opis implementacji}
    {\color{blue}
    Program został stworzony w języku Python 3.
    Klasa Board odpowiada za przechowywanie stanu planszy, związanych z nią informacji oraz
    przemieszczanie elementów planszy.
    Klasa Solver oraz klasy po niej dziedziczące zawierają logikę odpowiedzalną za rozwiązanie układanki daną metodyką.
    Klasa Logger odpowiada za zbieraniu dodatkowych informacji, może być połaczona z klasami rozwiązującymi
    korzystając ze wzorca projektowego obserwatora zaimplemetowanego poprzez klasy ObserverMixin oraz ObservableMixin.

    Powinien się tu również znaleźć diagram UML (diagram klas)


    Należy tu zamieścić krótki i zwięzły opis zaprojektowanych klas oraz powiązań
    między nimi.
        prezentujący najistotniejsze elementy stworzonej aplikacji. Należy także podać,




    }

    \section{Materiały i metody}
    {\color{blue}
    W tym miejscu należy opisać, jak przeprowadzone zostały wszystkie badania,
        których wyniki i dyskusja zamieszczane są w dalszych sekcjach. Opis ten
        powinien być na tyle dokładny, aby osoba czytająca go potrafiła wszystkie
    przeprowadzone badania samodzielnie powtórzyć w celu zweryfikowania ich
    poprawności. Przy opisie należy odwoływać się i stosować do
    opisanych w sekcji drugiej wzorów i oznaczeń, a także w jasny sposób opisać
    cel konkretnego testu. Najlepiej byłoby wyraźnie wyszczególnić (ponumerować)
        poszczególne eksperymenty tak, aby łatwo było się do nich odwoływać dalej.}

    \section{Wyniki}
    {\color{blue}
    W tej sekcji należy zaprezentować, dla każdego przeprowadzonego eksperymentu,
        kompletny zestaw wyników w postaci tabel, wykresów (preferowane) itp. Powinny
    być one tak ponazywane, aby było wiadomo, do czego się odnoszą. Wszystkie
    tabele i wykresy należy oczywiście opisać (opisać co jest na osiach, w
    kolumnach itd.) stosując się do przyjętych wcześniej oznaczeń. Nie należy tu
    komentować i interpretować wyników, gdyż miejsce na to jest w kolejnej sekcji.
    Tu również dobrze jest wprowadzić oznaczenia (tabel, wykresów), aby móc się do
    nich odwoływać poniżej.}

    \section{Dyskusja}
    {\color{blue}
    Sekcja ta powinna zawierać dokładną interpretację uzyskanych wyników
    eksperymentów wraz ze szczegółowymi wnioskami z nich płynącymi. Najcenniejsze
    są, rzecz jasna, wnioski o charakterze uniwersalnym, które mogą być istotne
    przy innych, podobnych zadaniach. Należy również omówić i wyjaśnić wszystkie
    napotkane problemy (jeśli takie były). Każdy wniosek powinien mieć poparcie we
    wcześniej przeprowadzonych eksperymentach (odwołania do konkretnych wyników).
    Jest to jedna z najważniejszych sekcji tego sprawozdania, gdyż prezentuje
    poziom zrozumienia badanego problemu.}

    \section{Wnioski}
    {\color{blue}
    W tej, przedostatniej, sekcji należy zamieścić podsumowanie najważniejszych
    wniosków z sekcji poprzedniej. Najlepiej jest je po prostu wypunktować. Znów,
        tak jak poprzednio, najistotniejsze są wnioski o charakterze uniwersalnym.}

    \begin{thebibliography}{0}
        \bibitem{l2short} T. Oetiker, H. Partl, I. Hyna, E. Schlegl.
        \textsl{Nie za krótkie wprowadzenie do systemu \LaTeX2e}, 2007, dost\k epny
        online.
    \end{thebibliography}

    {\color{blue}
    Na końcu należy obowiązkowo podać cytowaną w sprawozdaniu literaturę, z której
    grupa korzystała w trakcie prac nad zadaniem.}

\end{document}
